%You can leave alone everything before Line 79.
\documentclass{article}
\usepackage{url,amsfonts, amsmath, amssymb, amsthm,color, enumerate, graphicx}
% Page layout
\setlength{\textheight}{9.5in}
\setlength{\columnsep}{2.0pc}
\setlength{\textwidth}{6.5in}
\setlength{\topmargin}{-0.5in}
\setlength{\headheight}{0.0in}
\setlength{\headsep}{0.0in}
\setlength{\oddsidemargin}{0in}
\setlength{\evensidemargin}{0in}
\setlength{\parindent}{1pc}
\newcommand{\shortbar}{\begin{center}\rule{5ex}{0.1pt}\end{center}}
%\renewcommand{\baselinestretch}{1.1}
% Macros for course info
\newcommand{\courseNumber}{CSCI3383}
\newcommand{\courseTitle}{Algorithms}
\newcommand{\semester}{Spring 2019}
% Theorem-like structures are numbered within SECTION units
\theoremstyle{plain}
\newtheorem{theorem}{Theorem}[section]
\newtheorem{lemma}[theorem]{Lemma}
\newtheorem{corollary}[theorem]{Corollary}
\newtheorem{proposition}[theorem]{Proposition}
\newtheorem{statement}[theorem]{Statement}
\newtheorem{conjecture}[theorem]{Conjecture}
\newtheorem{fact}{Fact}
%definition style
\theoremstyle{definition}
\newtheorem{definition}[theorem]{Definition}
\newtheorem{example}{Example}
\newtheorem{problem}[theorem]{Problem}
\newtheorem{exercise}{Exercise}
\newtheorem{algorithm}{Algorithm}
%remark style
\theoremstyle{remark}
\newtheorem{remark}[theorem]{Remark}
\newtheorem{reduction}[theorem]{Reduction}
%\newtheorem{question}[theorem]{Question}
\newtheorem{question}{Question}
%\newtheorem{claim}[theorem]{Claim}
%
% Proof-making commands and environments
\newcommand{\beginproof}{\medskip\noindent{\bf Proof.~}}
\newcommand{\beginproofof}[1]{\medskip\noindent{\bf Proof of #1.~}}
\newcommand{\finishproof}{\hspace{0.2ex}\rule{1ex}{1ex}}
\newenvironment{solution}[1]{\medskip\noindent{\bf Problem #1.~}}{\shortbar}
%====header======
\newcommand{\solutions}[6]{
%\renewcommand{\thetheorem}{{#2}.\arabic{theorem}}
\vspace{-2ex}
\begin{center}
{\small  \courseNumber, \courseTitle
\hfill {\small \bf {#1} }\\
\semester \hfill
{\em Date: #3}}\\
\vspace{-1ex}
\hrulefill\\
\vspace{2ex}
{\LARGE Homework 4 Solutions}\\
\end{center}
\noindent
}
% math macros
\newcommand{\defeq}{\stackrel{\textrm{def}}{=}}
\newcommand{\Prob}{\textrm{Prob}}
%==
\begin{document}
%%%%%%%%%%%%%%%%%%%%%%%%%%%%%%%%%%%%%%%%%%%%%%%%%
\solutions{Will Farley, Courtney Peterson and Keith Carroll}{1}{\today}{}{}{}
%%%%%%%%%%%%%%%%%%%%%%%%%%%%%%%%%%%%%%%%%%%%%%%%%
%\renewcommand{\theproblem}{\arabic{problem}} 
%%%%%%%%%%%%%%%%%%%%%%%%%%%%%%%%%%%%%%%%%%%%%%%%%
%
% Begin the solution for each problem by
% \begin{solution}{Problem Number} and ends it with \end{solution}
%
% the solution for Problem 1

\begin{solution}{1}

The optimal solution uses a red-black tree.\\
A red-black tree yields a balanced BST.\\
From there a recursive function should dive down to each leaf then compare the differences between a leaf and its parent with the current highest difference.\\
The recursive function would output the greatest difference.\\
Searching red-black trees in this way takes O(lgn) time.\\\

\end{solution}

\begin{solution}{2.a}

See attached\\

\end{solution}

\begin{solution}{2.b}

See attached\\

\end{solution}

\begin{solution}{3}

-court

\end{solution}

\begin{solution}{4.a}

See attached.

\end{solution}

\begin{solution}{4.b}

Ask Su

\end{solution}

\begin{solution}{5.a}


\end{solution}

\begin{solution}{5.b}


\end{solution}

\begin{solution}{6.a}


\end{solution}

\begin{solution}{6.b}


\end{solution}

\begin{solution}{6.c}


\end{solution}


%
\end{document}
